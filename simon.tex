\documentclass[11pt]{book}
\usepackage[dvipsnames]{xcolor}
\usepackage{amssymb,latexsym}
\usepackage{graphicx}
\usepackage[spanish,mexico,es-nolayout]{babel}
\usepackage[utf8]{inputenc}
\usepackage{amsmath}
\usepackage{amssymb}
\usepackage{amsthm}
\usepackage{graphicx}
\usepackage{color}
\usepackage{tikz}
\usepackage{tkz-berge}

\usepackage{makeidx}


\newtheorem{theorem}{Teorema}
\newtheorem{corollary}{Corolario}
\newtheorem{proposition}{Proposición} 

\theoremstyle{definition}

\newtheorem{definition}{Definición}
\newtheorem{notation}{Notación}
\newtheorem{example}{Ejemplo}
\newtheorem{demostration}{Demostración}
\newcommand{\beao}{\begin{eqnarray*}} %Para centrar y no enumerar ecuaciones
\newcommand{\eeao}{\end{eqnarray*}}
\newcommand{\bea}{\begin{eqnarray}}   %Para centrar y enumerar ecuaciones
\newcommand{\eea}{\begin{eqnarray}}
\newcommand\N{{\mathbb N}}

\newcounter{in}
\newcounter{ini}

\makeindex






\begin{document}

\title{Title}
\author{Jesús Simón Mejía Cruz}
\date{}
\maketitle


\newpage
\thispagestyle{empty}
 \chapter*{Agradecimientos}

Probando git.



\newpage
 \thispagestyle{empty}





 \chapter{Introducción}

La grafic




\tableofcontents




\chapter{Conceptos básicos}

Aqui va una introduccion a los conceptos basicos


\section{Gráficas}

En esta sección se encuentran algunas definiciones necesarias de la teoria de gráficas.


\begin{definition}\textbf{Gráfica}\index{Gráficas}

Una gráfica $G$ es un par ordenado de conjuntos
$G=\{V(G),E(G)\}$ donde $V(G)$ es un conjunto
finito, llamado vértices de $G$, y

\begin{equation*}
 E(G) \subseteq
V(G)^{(2)}=\{e\subseteq V(G) \mid |e|=2\}
\end{equation*}
\end{definition}


\begin{definition}\textbf{Complemento de una gráfica}\\\index{Complemento de una gráfica}
Si G es una gráfica, definimos $\overline{G}$, el complemento de G
como

\begin{equation*}
V(\overline{G})=V(G)\\
E(\overline{G})=V(G)^{(2)}\setminus E(G)
\end{equation*}
\end{definition}



\begin{definition}\textbf{Orden de una gráfica}\\\index{Orden de una gráfica}
El orden $|G|$ de una gráfica $G$ es $|V(G)|$.
\end{definition}

\begin{definition}\textbf{Vértices adyacentes}\\\index{Vértices adyacentes}
Sea $G$ una gráfica, si $x , y \in V(G)$ tal que
$\{x,y\}\in E(G)$ diremos que $x$ y $y$ son adyacentes.
\end{definition}

En algunas ocasiones, aunque de manera no tan formal, también se les
conoce como vecinos o amigos.




\begin{definition}\textbf{Isomorfismo}\\\index{Isomorfismo}
Sean $G_1,G_2$ dos gráficas, un isomorfismo de
$G_1,G_2$ es una función biyectiva $f:
V(G_1)\rightarrow V(G_2)$ tal que $x\thicksim y$ en
$G_1$ si y solo si $f(x)\thicksim f(y)$ en $G_2$.
\end{definition}


\begin{definition}\textbf{Gráficas isomorfas}\\\index{Gráficas isomorfas}
Sean $G_1,G_2$ dos gráficas, si existe un isomorfismo
entre $G_1$ y $G_2$, decimos que $G_1$ y $G_2$
son isomorfas.
\end{definition}


\begin{definition}\textbf{Automorfismo}\\\index{Automorfismo}
Un isomorfismo de $G$ en $G$ se llama automorfismo de
$G$.
\end{definition}





\subsection{Familias de gráficas}


\begin{definition}\textbf{Gráfica completa}\\\index{Gráfica completa}
La gráfica con n vértices y todas las aristas posibles se llama
gráfica completa y se denota $K_n$ con $n\geq 1$.
\end{definition}



\begin{definition}\textbf{Unión disjunta}\\\index{Unión disjunta}
Dadas dos gráficas $G$ y $H$, definimos la unión disjunta $G\cup H$ como
$$V(G\cup H)=V(G)\dot{\cup}V(H)$$ $$E(G\cup H)=E(G)\dot{\cup}E(H)$$
\end{definition}

$G\bigcup G$ se denota por $2G$. En general $G\bigcup ...\bigcup G$
n veces, se denota $nG$. \index{nG}



\begin{definition}\textbf{Gráfica $nK_m$}\\\index{Gráfica nK_m}
A la unión disjunta de n gráficas $K_m$, se le denota por $nK_m$
\end{definition}


\begin{definition}\textbf{Subgráfica}\\\index{Subgráfica}
Una gráfica $H$ es subgráfica de $G$ si $V(H)\subseteq V(G)$ y
$E(H)\subseteq E(G)$.
\end{definition}


 

\subsection{Clanes}

\begin{definition}\textbf{Clan}\\\index{Clan}
Un clan es una subgráfica completa maximal.
\end{definition}

Es decir, es una subgráfica completa $H$, a la cual no se le puede
agregar ningún otro vértice, que no este en ella, con las
correspondientes aristas entre el y los vértices de esta, que haga que
la nueva gráfica sea completa.

\begin{example}
  Los clanes de la gráfica~\ref{grafclan}, son las subgráfica de la
  figura~\ref{clanes}. Tiene un clan isomorfo a $K_4$ y tres isomorfos
  a $K_3$.






\begin{figure}
\centering
 \begin{tikzpicture}
 \SetVertexNoLabel

\SetUpVertex[MinSize=1.5pt]
 \grCycle[RA=3,prefix=v]{8}

 \foreach \y in {0,1,...,7}{%
\setcounter{in}{\y} \stepcounter{in}
   \draw (v\y) node[below right]{$\alph{in}$};
}


 \Edges(v1,v3,v7,v1,v5,v3)
 \Edges(v5,v7)
 \end{tikzpicture}
 \caption{} \label{grafclan}
\end{figure}






\begin{figure}
\centering
 \begin{tikzpicture}
 \SetVertexNoLabel
\SetUpVertex[MinSize=1.5pt]

 \grEmptyPath*[x=0,y=0,RA=2,prefix=v]{,,,}
 \grEmptyPath*[x=1,y=2,RA=4,prefix=w]{,}
 \grEmptyPath*[x=0,y=3,RA=2,prefix=x]{,,,}
 \grEmptyPath*[x=0,y=5,RA=2,prefix=y]{,}
 \Vertex[x=5,y=5]{z0}



%\AssignVertexLabel {v}{$v_1$,$v_2$}
% \Edges(v0,v1)
 %\Edges[color=Brown](a3;0,v0,a3;4)

 \Edges(v0,v1,w0,v0)
 \Edges(v2,v3,w1,v2)
 \Edges(x0,x1,y1,x0,y0,x1)
 \Edges(y0,y1)
 \Edges(x2,x3,z0,x2)
% \Edges[style={bend left=30},color=blue](a3;4,a3;8)


\end{tikzpicture}
\caption{Clanes de la gráfica\ref{grafclan}}\label{clanes}
\end{figure}


\end{example}



\begin{definition}\textbf{Gráfica de Clanes de H}\\\index{Gráfica de Clanes de H}
La gráfica de clanes de $H$ es la gráfica $K(G)$,tal que: 
\begin{enumerate}
\item $V(K(G)=\{Q\subseteq V(G)$ tal que $Q$ es clan de $ G\}$ 
\item $E(K(G)=\{\{Q_1,Q_2\}$ tal que $Q_1 \ne Q_2,Q_1\cap Q_2 \ne\emptyset\} $ 
\end{enumerate}
entonces a $K(G)$ se le conoce como gráfica de clanes de $G$.
\end{definition}




\begin{definition}\textbf{Gráficas iteradas de clanes}\\\index{Gráficas iteradas de clanes}
Las Gráficas itaradas de clanes se definen como:
 
\begin{enumerate}
\item $k^0(G)=G$, $k^n(G)=k(k^{n-1}(G))$, $n\ge 1$
\end{enumerate}

\end{definition}

\begin{definition}\textbf{Bipartita}\\\index{Bipartita}
Sea $B$ una gráfica, si existen conjuntos $A$ y $B$ de $V(B)$ tales que 
\begin{enumerate}
\item $V(A)=A\cup B$ y $A\cap B=\empty$ 
\item Si $x\sim y$ entonces $x\in A$ y $y\in B$ o viceversa
\end{enumerate}
entonces a $B$ se le conoce como gráfica bipartita.
\end{definition}

En este caso, se dice que los conjuntos $A$ y $B$ son una bipartición\index{Bipartición} de $B$ y se denota por $B(X,Y)$. Además los vértices de $B$ se pueden colorear con dos colores, sin que una arista tenga el mismo color en ambos extremos. Si los vértices en $A$ se colorean con el color uno, y los de $B$ con el dos, se tiene lo deseado, pues no existe arista entre dos elementos del mismo conjunto.



\begin{definition}\textbf{Bipartita completa}\\\index{Bipartita completa}
Dados $m,n\geq 1$, la gráfica bipartita completa $K_{m,n}$ se define
como $K_{m,n}=\overline{K_m \cup K_n}$
\end{definition}


\begin{definition}\textbf{Bipartita clánica}\\\index{Bipartita clánica}
Sea $Cl_G:=\{H \mid H$ es un clan de una gráfica $G \}$. Se define la gráfica bipartita clánica $BC_G$, de una gráfica $G$, como $V(BC_G)=V(G)\dot{\cup}Cl_G$ donde $x,y\in V(BC_G)$ son adyacentes si y solo si $x\in V(G)$ y $y$ es un clan de $G$ tales que $x\in y$ o viceversa.   
\end{definition}

En otras palabras, la gráfica bipartita clánica es aquella cuyos vértices están asociados a los elementos de $V(G)$ y a los clanes de $G$,de manera disjunta.Donde las aristas que existen se dan solo entre un elemento asociado a un vértices y otro asociado a un clan. La figura~\ref{BCG} es la gráfica bipartita clánica de la gráfica~\ref{grafclan}.





\begin{figure}
\centering
 \begin{tikzpicture}
 \SetVertexNoLabel
\SetUpVertex[MinSize=1.5pt]

 \grEmptyPath*[x=0,y=3,RA=2,prefix=v]{,,,,,,,}
 \grEmptyPath*[x=1,y=0,RA=3,prefix=w]{,,,,}



 \foreach \y in {0,1,...,7}{%
\setcounter{in}{\y} \stepcounter{in}
   \draw (v\y) node[above right]{$\alph{in}$};
}

 \Edges(v0,w0,v1,w0,v7)
 \Edges(v1,w1,v2,w1,v3)
 \Edges(v3,w3,v4,w3,v5)

 \Edges(v5,w4,v6,w4,v7)

 \Edges(v1,w2,v3,w2,v5,w2,v7)
% \Edges[style={bend left=30},color=blue](a3;4,a3;8)
\draw(1,-.5) node {$K_3$};
   \draw(4,-.5) node {$K_3$};
    \draw(7,-.5) node {$K_4$};
     \draw(10,-.5) node {$K_3$};
  \draw(13,-.5) node {$K_3$};

\end{tikzpicture}
\caption{Gráfica bipartita clánica de \ref{grafclan}}\label{BCG}
\end{figure}






\begin{definition}\textbf{Subgráfica inducida}\\\index{Subgráfica inducida}
La subgráfica $H$ de $G$ se dice inducida si $\{x,y\}\in V(H)^{(2)}\cap
E(G)$ entonces $\{x,y\}\in E(H)$.
\end{definition}

Una subgráfica inducida esta determinada por su conjunto de
vértices. Si $X\in V(G)$, la subgráfica de $G$ inducida por $X$ se
denota por $G[X]$.

\begin{proposition}\textbf{}\\
Si $G$ y $H$ son gráficas isomorfas, y la gráfica $L$ es subgráfica
de $G$ entonces $H$ tiene una subgráfica isomorfa a $L$.
\end{proposition}

Demostración: Sea $\varphi : G \rightarrow H$ un isomorfismo entre G
y H, una subgráfica de $H$ isomorfa a $L$ es la gráfica inducida por
$\varphi (V(L))$, si $x\sim y$ en $L\subset G$ entonces $\varphi
(x)\sim \varphi (y)$ en H.





\chapter{Conclusiones}

AQUI VAN LAS CONCLUSIONES

\begin{thebibliography}{0}


\bibitem{Ehrlich}
Ehrlich, Gertrude. \emph{ Fundamental Concepts of Abstract Algebra}. PWS-KENT Publishing Company, 1991.

\bibitem{Barrera}
Barrera Mora, Fernando. \emph{ Introduccion a la Teoria de Grupos}. UAEH-SMM, septiembre 2004.

\bibitem{Harary}
F. Harary, \emph{ Graph Theory}, Addison-Wesley, Reading, MA. 1969.

\bibitem{9jaula}
Gunnar Brinkmann, Brenda D. McKay, and Carsten Saager. \emph{ The smallest cubic graf of girth nine}. 

\bibitem{cuellogrande}
Biggs, Norman. \emph{Construction for cubic graph with large girth}. Agosto 1998. 

\bibitem{survey}
Wong, Pak-Ken . \emph{Cage-A survey}. Journal of Graph Theory, Vol. 6 (1982) 1-22.

\end{thebibliography}



\printindex



\end{document}
